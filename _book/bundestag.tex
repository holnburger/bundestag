\PassOptionsToPackage{unicode=true}{hyperref} % options for packages loaded elsewhere
\PassOptionsToPackage{hyphens}{url}
%
\documentclass[oneside, 12pt, numbers=endperiod]{scrbook}
\usepackage{lmodern}
\usepackage{setspace}
\setstretch{1.5}
\usepackage{amssymb,amsmath}
\usepackage{ifxetex,ifluatex}
\usepackage{fixltx2e} % provides \textsubscript
\ifnum 0\ifxetex 1\fi\ifluatex 1\fi=0 % if pdftex
  \usepackage[T1]{fontenc}
  \usepackage[utf8]{inputenc}
  \usepackage{textcomp} % provides euro and other symbols
\else % if luatex or xelatex
  \usepackage{unicode-math}
  \defaultfontfeatures{Ligatures=TeX,Scale=MatchLowercase}
\fi
% use upquote if available, for straight quotes in verbatim environments
\IfFileExists{upquote.sty}{\usepackage{upquote}}{}
% use microtype if available
\IfFileExists{microtype.sty}{%
\usepackage[]{microtype}
\UseMicrotypeSet[protrusion]{basicmath} % disable protrusion for tt fonts
}{}
\IfFileExists{parskip.sty}{%
\usepackage{parskip}
}{% else
\setlength{\parindent}{0pt}
\setlength{\parskip}{6pt plus 2pt minus 1pt}
}
\usepackage{hyperref}
\hypersetup{
            pdftitle={Geschlechterunterschiede im Deutschen Bundestag},
            pdfauthor={Josef Holnburger und Gina-Gabriela Görner},
            pdfborder={0 0 0},
            breaklinks=true}
\urlstyle{same}  % don't use monospace font for urls
\usepackage{color}
\usepackage{fancyvrb}
\newcommand{\VerbBar}{|}
\newcommand{\VERB}{\Verb[commandchars=\\\{\}]}
\DefineVerbatimEnvironment{Highlighting}{Verbatim}{commandchars=\\\{\}}
% Add ',fontsize=\small' for more characters per line
\usepackage{framed}
\definecolor{shadecolor}{RGB}{248,248,248}
\newenvironment{Shaded}{\begin{snugshade}}{\end{snugshade}}
\newcommand{\AlertTok}[1]{\textcolor[rgb]{0.94,0.16,0.16}{#1}}
\newcommand{\AnnotationTok}[1]{\textcolor[rgb]{0.56,0.35,0.01}{\textbf{\textit{#1}}}}
\newcommand{\AttributeTok}[1]{\textcolor[rgb]{0.77,0.63,0.00}{#1}}
\newcommand{\BaseNTok}[1]{\textcolor[rgb]{0.00,0.00,0.81}{#1}}
\newcommand{\BuiltInTok}[1]{#1}
\newcommand{\CharTok}[1]{\textcolor[rgb]{0.31,0.60,0.02}{#1}}
\newcommand{\CommentTok}[1]{\textcolor[rgb]{0.56,0.35,0.01}{\textit{#1}}}
\newcommand{\CommentVarTok}[1]{\textcolor[rgb]{0.56,0.35,0.01}{\textbf{\textit{#1}}}}
\newcommand{\ConstantTok}[1]{\textcolor[rgb]{0.00,0.00,0.00}{#1}}
\newcommand{\ControlFlowTok}[1]{\textcolor[rgb]{0.13,0.29,0.53}{\textbf{#1}}}
\newcommand{\DataTypeTok}[1]{\textcolor[rgb]{0.13,0.29,0.53}{#1}}
\newcommand{\DecValTok}[1]{\textcolor[rgb]{0.00,0.00,0.81}{#1}}
\newcommand{\DocumentationTok}[1]{\textcolor[rgb]{0.56,0.35,0.01}{\textbf{\textit{#1}}}}
\newcommand{\ErrorTok}[1]{\textcolor[rgb]{0.64,0.00,0.00}{\textbf{#1}}}
\newcommand{\ExtensionTok}[1]{#1}
\newcommand{\FloatTok}[1]{\textcolor[rgb]{0.00,0.00,0.81}{#1}}
\newcommand{\FunctionTok}[1]{\textcolor[rgb]{0.00,0.00,0.00}{#1}}
\newcommand{\ImportTok}[1]{#1}
\newcommand{\InformationTok}[1]{\textcolor[rgb]{0.56,0.35,0.01}{\textbf{\textit{#1}}}}
\newcommand{\KeywordTok}[1]{\textcolor[rgb]{0.13,0.29,0.53}{\textbf{#1}}}
\newcommand{\NormalTok}[1]{#1}
\newcommand{\OperatorTok}[1]{\textcolor[rgb]{0.81,0.36,0.00}{\textbf{#1}}}
\newcommand{\OtherTok}[1]{\textcolor[rgb]{0.56,0.35,0.01}{#1}}
\newcommand{\PreprocessorTok}[1]{\textcolor[rgb]{0.56,0.35,0.01}{\textit{#1}}}
\newcommand{\RegionMarkerTok}[1]{#1}
\newcommand{\SpecialCharTok}[1]{\textcolor[rgb]{0.00,0.00,0.00}{#1}}
\newcommand{\SpecialStringTok}[1]{\textcolor[rgb]{0.31,0.60,0.02}{#1}}
\newcommand{\StringTok}[1]{\textcolor[rgb]{0.31,0.60,0.02}{#1}}
\newcommand{\VariableTok}[1]{\textcolor[rgb]{0.00,0.00,0.00}{#1}}
\newcommand{\VerbatimStringTok}[1]{\textcolor[rgb]{0.31,0.60,0.02}{#1}}
\newcommand{\WarningTok}[1]{\textcolor[rgb]{0.56,0.35,0.01}{\textbf{\textit{#1}}}}
\usepackage{longtable,booktabs}
% Fix footnotes in tables (requires footnote package)
\IfFileExists{footnote.sty}{\usepackage{footnote}\makesavenoteenv{longtable}}{}
\usepackage{graphicx,grffile}
\makeatletter
\def\maxwidth{\ifdim\Gin@nat@width>\linewidth\linewidth\else\Gin@nat@width\fi}
\def\maxheight{\ifdim\Gin@nat@height>\textheight\textheight\else\Gin@nat@height\fi}
\makeatother
% Scale images if necessary, so that they will not overflow the page
% margins by default, and it is still possible to overwrite the defaults
% using explicit options in \includegraphics[width, height, ...]{}
\setkeys{Gin}{width=\maxwidth,height=\maxheight,keepaspectratio}
\setlength{\emergencystretch}{3em}  % prevent overfull lines
\providecommand{\tightlist}{%
  \setlength{\itemsep}{0pt}\setlength{\parskip}{0pt}}
\setcounter{secnumdepth}{5}
% Redefines (sub)paragraphs to behave more like sections
\ifx\paragraph\undefined\else
\let\oldparagraph\paragraph
\renewcommand{\paragraph}[1]{\oldparagraph{#1}\mbox{}}
\fi
\ifx\subparagraph\undefined\else
\let\oldsubparagraph\subparagraph
\renewcommand{\subparagraph}[1]{\oldsubparagraph{#1}\mbox{}}
\fi

% set default figure placement to htbp
\makeatletter
\def\fps@figure{htbp}
\makeatother

\usepackage{booktabs}
\usepackage[ngerman]{babel}
\usepackage[utf8]{inputenc}

\setkomafont{disposition}{\normalcolor\bfseries}

% Style - Seitennummern oben, Linie oben, keine Linien bei Kapiteln, Kapitel nicht abgesetzt
\usepackage{scrlayer-scrpage}
\automark{chapter}
\automark{section}
\clearpairofpagestyles
\ihead{\headmark}
\ohead*{\pagemark}
\KOMAoptions{headsepline = true}

% Wir machen die Chapterheads etwas kleiner, bzw. verringern den Abstand nach oben und unten.
\renewcommand*\chapterheadstartvskip{\vspace*{-\topskip}}
\renewcommand*\chapterheadendvskip{%
  \vspace*{1\baselineskip plus .1\baselineskip minus .167\baselineskip}}

% Mit folgendem Befehl wird keine Titleseite gesetzt, wir haben eine eigene
\AtBeginDocument{\let\maketitle\relax}

% Hintergrund für den Code
\definecolor{lightgrey}{rgb}{0.90,0.90,0.90}

\setlength{\textwidth}{16cm}        % Textbreite
\setlength{\textheight}{24cm}       % Texthöhe
\setlength{\topmargin}{-12mm}       % oberer Rand
\newcommand\myName{Josef Holnburger}
\newcommand\myStreetAddress{XXX}
\newcommand\myCityAddress{XXX}
\newcommand\myEmail{josef@holnburger.com}
\newcommand\myKeywords{} % Optional
\newcommand\myMatNr{XXX}
\newcommand\myTitle{Geschlechterunterschiede im Deutschen Bundestag}
\newcommand\mySubTitle{Arbeitstitel}
\newcommand\thesisType{Forschungsarbeit}
\newcommand\fachbereich{Sozialwissenschaften} 
\newcommand\fachgebiet{Politikwissenschaft}
\newcommand\courseOfStudies{Musterseminar}
\newcommand\supervisorType{Dozenten} % Dozent*in, Seminarleiter etc.
\newcommand\supervisor{Prof. Dr. Kai-Uwe Schnapp \\ PD Dr. Falk Daviter}
\newcommand\currentSemester{Wintersemester 2018/19}
\newcommand\dateOfSubmission{tba}

\hypersetup
  {
  	pdftitle     = \myTitle,
  	pdfsubject   = {Universität Hamburg, \thesisType},
  	pdfauthor    = \myName,
  	pdfkeywords  = \myKeywords,
  	plainpages   = false, pdfpagelabels
  }

\title{Geschlechterunterschiede im Deutschen Bundestag}
\author{Josef Holnburger und Gina-Gabriela Görner}
\date{}

\usepackage{amsthm}
\newtheorem{theorem}{Theorem}[chapter]
\newtheorem{lemma}{Lemma}[chapter]
\theoremstyle{definition}
\newtheorem{definition}{Definition}[chapter]
\newtheorem{corollary}{Corollary}[chapter]
\newtheorem{proposition}{Proposition}[chapter]
\theoremstyle{definition}
\newtheorem{example}{Example}[chapter]
\theoremstyle{definition}
\newtheorem{exercise}{Exercise}[chapter]
\theoremstyle{remark}
\newtheorem*{remark}{Remark}
\newtheorem*{solution}{Solution}
\begin{document}
\maketitle

% *************************************************************************
% *    Thesis / Dissertation Latex Template                                        
% *    
% *    Author: Leonard Heilig <leonard.heilig@uni-hamburg.de>
% *    modified by: Josef Holnburger <josef@holnburger.com>
% *   
% *    Note: some parts of this template are based on the VSIS template
% *              of Michael von Riegen <riegen@informatik.uni-hamburg.de>
% *   
% *************************************************************************

\begin{titlepage}

% START PAGE: -1
\setcounter{page}{-1}    

\begin{figure}[h]
	\begin{minipage}[t]{8.5cm}
	\flushleft 
			%Presented by: \> \textbf{\myName} \\
			%\> \textrm{\myAddress} \\
			%\> \textrm{\myEmail} \\
			Universität Hamburg \\
			Fachbereich: \fachbereich \\
			Fachgebiet: \fachgebiet \\
			Seminar: \courseOfStudies \\ 
			\supervisorType: \supervisor \\
			\currentSemester \\
	\end{minipage}
	\hfill
   \begin{minipage}[t][2cm][b]{0.4\textwidth}
    \flushright\noindent
    	 \noindent\includegraphics{images/UHH-Logo_2010_Farbe_CMYK.pdf}
    \end{minipage}
\end{figure}

\vspace*{\fill}
\begin{center}
	% THESIS TYPE
	\vspace{1cm}\noindent {\textbf{\thesisType}} \vspace{0.2cm} \\
	% THESIS TITLE
	\textbf{\Large \myTitle} \\
	\textbf{\mySubTitle} \\
	\dateOfSubmission \\
\end{center}
\vspace*{\fill}

\vspace*{\fill}%
\begin{figure}
	\myName \\
	Matrikelnummer:  \myMatNr \\
	\myStreetAddress \vspace{0.1cm} \\ 
	\myCityAddress \vspace{0.1cm}  \\
	E-Mail: \myEmail \\ 
\end{figure}

\end{titlepage}

{
\setcounter{tocdepth}{1}
\tableofcontents
}
\frontmatter
\pagenumbering{Roman}

\listoffigures
\addcontentsline{toc}{chapter}{\listfigurename}
\vspace*{24pt}
{\let\clearpage\relax \listoftables}    
\addcontentsline{toc}{chapter}{\listtablename}

\mainmatter

\hypertarget{intro}{%
\chapter{Einleitung}\label{intro}}

Zusammen mit Gina-Gabriela Görner analysiere ich die Protokolle des
Deutschen Bundestages auf mögliche Geschlechterunterschiede. Hierfür
wollen wir die Anzahl aber auch Inhalte der Reden mehrer Wahlperioden
des Bundestags untersuchen. Dieses Projekt wurde auch für das
\href{http://symposium.computationalsocialscience.eu/2018/}{European
Symposium Series on Societal Challenges} eingereicht und wir dürfen es
dort mit einem Plakat vorstellen.

Wir orientieren uns vor allem an der Forschung von Bäck et al.
(\protect\hyperlink{ref-back_2014}{2014}), welche das schwedische
Parlament auf mögliche Geschlechterunterschiede und Diskriminierung hin
untersucht haben. In dieser Studie wurden Unterschiede sowohl in der
Anzahl als auch bezüglich des Inhalts der Reden festgestellt. Auuch im
schwedischen Parlament sind Männer deutlich häufiger zu hören -- obwohl
es mit einem Frauenanteil von 40 Prozent die höchste Quote europäischer
Parlamente aufweist (Bäck et al.
\protect\hyperlink{ref-back_2014}{2014}: 505). Männer sprechen in ihren
Reden häufiger über \emph{hard topics}, bei \emph{soft topics} ist der
Redeanteil hingegen ausgeglichen (ebd: 513ff.). Die Konstruktion der
\emph{hard} und \emph{soft topics} geht dabei auf Wangnerud
(\protect\hyperlink{ref-wangnerud_2000}{2000}) zurück und ist nicht
unkritisch -- hier werden durchaus Geschlechterstereotype
aufrechterhalten oder gar reproduziert, indem ``typische'' Frauen und
Männerthemen identifiziert werden. Wangnerud hat ihn ihrer Untersuchung
die Mitglieder des schwedischen \emph{Riksdag} bezüglich ihrer
Aktivitäten befragt. Das Ergebnis von Bäck et al.
(\protect\hyperlink{ref-back_2014}{2014}) ist deshalb auch nicht
besonders überraschend -- bestätigt es doch nur, dass die
Fachpolitiker\_innen häufiger über ihre Themen auch im Plenum reden.

In dieser Arbeit soll anders vorgegangen werden. Die Inhalte der Reden
im Bundestaug sollen ohne voherige Identifikation vermeintlicher Frauen-
und Männerthemen untersucht werden. Hierbei nutzen wir die Möglichkeiten
des \emph{Topic Modelling} um zunächst generell Themen der Reden im
Bundestags zu identifizieren und diese anschließend auf mögliche
Geschlechterunterschiede untersuchen. Natürlich wollen auch wir die
Unterschiedlichen Redeanteile untersuchen.

Da ich den Prinzipien der Open Science sehr viel abgewinnen kann, soll
die Erhebung und Auswertung möglichst transparent und nachvollziehbar
dargestellt werden.

\hypertarget{data-01}{%
\chapter{Generelles zur Datenerhebung}\label{data-01}}

Die Reden im Deutschen Bundestag sind in den Protokollen dokumentiert
und lassen sich \href{https://www.bundestag.de/protokolle}{online
abrufen}. Praktischerweise liegen die Daten seit der aktuellen
Wahlerperiode auch im \href{http://www.tei-c.org/}{TEI-Format (Text
Encoding Initiative)} vor. Dies erleichtert die Analyse der Protkolle
erheblich. Die Datenerhebung und Auswertung erfolgt mit der
Programmiersprache R (R Core Team
\protect\hyperlink{ref-rcoreteam_2018}{2018}) und der \texttt{tidyverse}
Packetsammlung (Wickham \protect\hyperlink{ref-wickham_2017}{2017}).

Die Datenerhebung soll zunächst an einem Beispielprotokoll gezeigt
werden -- wir nutzen hierfür das Protokoll der
\href{https://www.bundestag.de/blob/577958/b2d1fce9b7dec32a1403a2ec5f6bc58d/19061-data.xml}{61.
Sitzung des 19. Bundestages}. Das Protokoll liegt dabei sowohl als PDF,
als TXT und auch als XML-Datei vor. Letzeres wird für diese Arbeit
herangezogen.

Mit dem Packet \texttt{xml2} (Wickham et al.
\protect\hyperlink{ref-wickham_2018}{2018}) kann das Protokoll
ausgelesen und anschließend in ein passendes Format umgewandelt werden.
Mit der Funktion \texttt{read\_html()} wird das vollständige Protokoll
in der Variable \texttt{prot\_file} eingelesen. Die Umwandlung der
einzelnen Knoten und Attribute des XML-Dokuments erfolgt mit dem
\texttt{rvest} Packet (Wickham
\protect\hyperlink{ref-wickham_2016}{2016}).

Da für diese Auswertung nur die Reden im Deutschen Bundestag
herangezogen werden (und angehängte Dokumente sowie Anwesenheitslisten
irrelevant sind), soll nur ein Teil des Protkolls untersucht werden.
Mittels der Funktion \texttt{xml\_find\_all("//rede")} können alle
Einträge unter dem Knoten ``rede'' herausgefiltert werden.

\begin{Shaded}
\begin{Highlighting}[]
\KeywordTok{library}\NormalTok{(tidyverse)}
\KeywordTok{library}\NormalTok{(xml2)}
\KeywordTok{library}\NormalTok{(rvest)}

\NormalTok{prot_file <-}\StringTok{ }\KeywordTok{read_html}\NormalTok{(}\StringTok{"https://www.bundestag.de/blob/577958/b2d1fce9b7dec32a1403a2ec5f6bc58d/19061-data.xml"}\NormalTok{) }

\NormalTok{prot_overview <-}\StringTok{ }\NormalTok{prot_file }\OperatorTok
\StringTok{  }\KeywordTok{xml_find_all}\NormalTok{(}\StringTok{"//rede"}\NormalTok{)}
\end{Highlighting}
\end{Shaded}

Die Datei soll anschließend in einen \emph{Dataframe} umgewandelt
werden. Dies erleichtert die weitere Arbeit und im weiteren Verlauf
können die Daten einfacher nach nach Geschlecht, Partei, Datum oder
Wahlperiode gefiltert werden. Hierbei wird vor allem mit den Funktionen
\texttt{xml\_node()} und \texttt{xml\_attr()} gearbeitet. Zum
Verständnis bietet sich hier ein kleiner Diskurs an.

\hypertarget{xml-knoten-und-attribute}{%
\section{XML-Knoten und Attribute}\label{xml-knoten-und-attribute}}

Nachdem die Datei eingelesen wurde, lohnt sich ein Blick auf die
Rohdaten:

\begin{Shaded}
\begin{Highlighting}[]
\KeywordTok{<rede}\OtherTok{ id=}\StringTok{"ID196105400"}\KeywordTok{>}
  \KeywordTok{<p}\OtherTok{ klasse=}\StringTok{"redner"}\KeywordTok{><redner}\OtherTok{ id=}\StringTok{"11004826"}\KeywordTok{><name><vorname>}\NormalTok{Siemtje}\KeywordTok{</vorname><nachname>}\NormalTok{Möller}\KeywordTok{</nachname><fraktion>}\NormalTok{SPD}\KeywordTok{</fraktion></name></redner>}\NormalTok{Siemtje Möller (SPD):}\KeywordTok{</p>}
  \KeywordTok{<p}\OtherTok{ klasse=}\StringTok{"J_1"}\KeywordTok{>}\NormalTok{Lieber Jürgen Trittin, Sie haben gerade eigentlich das wiederholt, was Sie schon in Ihrem Redebeitrag gesagt haben. Ich möchte dazu genau dasselbe sagen, was ich auch in der Debatte über Syrien und eine mögliche Beteiligung der Bundeswehr gesagt habe: Wir sind da noch nicht. Es gilt, alles zu tun, damit dieser Zustand nicht eintritt. Und genau das tut Heiko Maas.}\KeywordTok{</p>}
  \KeywordTok{<kommentar>}\NormalTok{(Beifall bei der SPD)}\KeywordTok{</kommentar>}
\KeywordTok{</rede>}
\end{Highlighting}
\end{Shaded}

In diesem Beispiel wird das XML-Fragment in die Variable
\texttt{xml\_example} geladen und ausgewertet. Die Knoten eines
XML-Documents werden durch \texttt{\textless{}\textgreater{}} und
\texttt{\textless{}/\textgreater{}} eingefasst. Beispielsweise können
die Knoten mit den Namen ``Kommentar'' folgendermaßen extrahiert werden:

\begin{Shaded}
\begin{Highlighting}[]
\NormalTok{xml_example }\OperatorTok\StringTok{ }\KeywordTok{xml_nodes}\NormalTok{(}\StringTok{"kommentar"}\NormalTok{)}
\end{Highlighting}
\end{Shaded}

\begin{verbatim}
## {xml_nodeset (1)}
## [1] <kommentar>(Beifall bei der SPD)</kommentar>
\end{verbatim}

Bei der Ausgabe fällt jedoch auf, dass die Datei weiterhin eine
XML-Datei bleibt und die Knoteninformationen ebenfalls extrahiert
werden. Mittels der Funktion \texttt{xml\_text()} kann das Ergebniss in
einen in einen Character-String umwandelt werden.

\begin{Shaded}
\begin{Highlighting}[]
\NormalTok{xml_example }\OperatorTok\StringTok{ }\KeywordTok{xml_nodes}\NormalTok{(}\StringTok{"kommentar"}\NormalTok{) }\OperatorTok\StringTok{ }\KeywordTok{xml_text}\NormalTok{()}
\end{Highlighting}
\end{Shaded}

\begin{verbatim}
## [1] "(Beifall bei der SPD)"
\end{verbatim}

Die Ergebnisse werden in einer Liste zusammengefasst und können
beispielsweise in einem Datenframe umgewandelt werden.

Die Attribute eines Knotens finden sich in den Klammern nach dem
Gleichheitszeichen:
\texttt{\textless{}knotenname\ attribut="inhalt"\textgreater{}...}. Die
Werte eines Attributes (und auch den Attributnamen) können mit der
Funktion \texttt{xml\_attr()} extrahiert werden.

\begin{Shaded}
\begin{Highlighting}[]
\NormalTok{xml_example }\OperatorTok\StringTok{ }\KeywordTok{xml_node}\NormalTok{(}\StringTok{"rede"}\NormalTok{) }\OperatorTok\StringTok{ }\KeywordTok{xml_attr}\NormalTok{(}\StringTok{"id"}\NormalTok{)}
\end{Highlighting}
\end{Shaded}

\begin{verbatim}
## [1] "ID196105400"
\end{verbatim}

Mit dieser kurzen Exkursion können wir nun eine Funktion bauen, welche
auf die für uns relevanten Daten aus dem XML-Dokument extrahiert und
anschließend in einen Datenframe umwandelt.

\begin{Shaded}
\begin{Highlighting}[]
\NormalTok{get_overview_df <-}\StringTok{ }\ControlFlowTok{function}\NormalTok{(x)\{}
\NormalTok{  rede_id <-}\StringTok{ }\NormalTok{x }\OperatorTok\StringTok{ }\KeywordTok{xml_attr}\NormalTok{(}\StringTok{"id"}\NormalTok{)}
\NormalTok{  redner_id <-}\StringTok{ }\NormalTok{x }\OperatorTok\StringTok{ }\KeywordTok{xml_node}\NormalTok{(}\StringTok{"redner"}\NormalTok{) }\OperatorTok\StringTok{ }\KeywordTok{xml_attr}\NormalTok{(}\StringTok{"id"}\NormalTok{)}
\NormalTok{  redner_vorname <-}\StringTok{ }\NormalTok{x }\OperatorTok\StringTok{ }\KeywordTok{xml_node}\NormalTok{(}\StringTok{"redner"}\NormalTok{) }\OperatorTok\StringTok{ }\KeywordTok{xml_node}\NormalTok{(}\StringTok{"vorname"}\NormalTok{) }\OperatorTok\StringTok{ }\KeywordTok{xml_text}\NormalTok{()}
\NormalTok{  redner_nachname <-}\StringTok{ }\NormalTok{x }\OperatorTok\StringTok{ }\KeywordTok{xml_node}\NormalTok{(}\StringTok{"redner"}\NormalTok{) }\OperatorTok\StringTok{ }\KeywordTok{xml_node}\NormalTok{(}\StringTok{"nachname"}\NormalTok{) }\OperatorTok\StringTok{ }\KeywordTok{xml_text}\NormalTok{()}
\NormalTok{  redner_fraktion <-}\StringTok{ }\NormalTok{x }\OperatorTok\StringTok{ }\KeywordTok{xml_node}\NormalTok{(}\StringTok{"redner"}\NormalTok{) }\OperatorTok\StringTok{ }\KeywordTok{xml_node}\NormalTok{(}\StringTok{"fraktion"}\NormalTok{) }\OperatorTok\StringTok{ }\KeywordTok{xml_text}\NormalTok{()}
\NormalTok{  redner_rolle <-}\StringTok{ }\NormalTok{x }\OperatorTok\StringTok{ }\KeywordTok{xml_node}\NormalTok{(}\StringTok{"rolle_kurz"}\NormalTok{) }\OperatorTok\StringTok{ }\KeywordTok{xml_text}\NormalTok{()}
  
  \KeywordTok{data_frame}\NormalTok{(rede_id, redner_id, redner_vorname, redner_nachname, redner_fraktion, redner_rolle)}
\NormalTok{\}}
\end{Highlighting}
\end{Shaded}

Wir können mit dieser Funktion nun die vorher eingelesen XML-Datei in
einen Datenframe umwandeln und auswerten.

\begin{Shaded}
\begin{Highlighting}[]
\NormalTok{overview_df <-}\StringTok{ }\KeywordTok{get_overview_df}\NormalTok{(prot_overview)}

\NormalTok{overview_df}
\end{Highlighting}
\end{Shaded}

\begin{verbatim}
## # A tibble: 144 x 6
##    rede_id redner_id redner_vorname redner_nachname redner_fraktion
##    <chr>   <chr>     <chr>          <chr>           <chr>          
##  1 ID1961~ 11003196  Andrea         Nahles          SPD            
##  2 ID1961~ 11004873  Ulrike         Schielke-Ziesi~ AfD            
##  3 ID1961~ 11002666  Hermann        Gröhe           CDU/CSU        
##  4 ID1961~ 11004179  Johannes       Vogel           FDP            
##  5 ID1961~ 11004012  Matthias W.    Birkwald        DIE LINKE      
##  6 ID1961~ 11003578  Markus         Kurth           BÜNDNIS 90/DIE~
##  7 ID1961~ 11003142  Hubertus       Heil            <NA>           
##  8 ID1961~ 11004856  Jürgen         Pohl            AfD            
##  9 ID1961~ 11002812  Max            Straubinger     CDU/CSU        
## 10 ID1961~ 11004941  Gyde           Jensen          FDP            
## # ... with 134 more rows, and 1 more variable: redner_rolle <chr>
\end{verbatim}

\hypertarget{zwischenfazit}{%
\section{Zwischenfazit}\label{zwischenfazit}}

Wir konnten mit wenigen Zeilen Code das XML-Format in einen Datenframe
umwandeln, welcher uns die weitere Arbeit erheblich erleichter. So
könnten wir sehr schnell sagen, wie viele Reden es von den einzelnen
Fraktionen zur 61. Sitzung des 19. Bundestags gab:

\begin{Shaded}
\begin{Highlighting}[]
\NormalTok{overview_df }\OperatorTok\StringTok{ }
\StringTok{  }\KeywordTok{group_by}\NormalTok{(redner_fraktion) }\OperatorTok
\StringTok{  }\KeywordTok{summarise}\NormalTok{(}\DataTypeTok{reden =} \KeywordTok{n}\NormalTok{()) }\OperatorTok
\StringTok{  }\KeywordTok{arrange}\NormalTok{(}\OperatorTok{-}\NormalTok{reden)}
\end{Highlighting}
\end{Shaded}

\begin{verbatim}
## # A tibble: 8 x 2
##   redner_fraktion       reden
##   <chr>                 <int>
## 1 CDU/CSU                  41
## 2 SPD                      27
## 3 AfD                      20
## 4 BÜNDNIS 90/DIE GRÜNEN    16
## 5 FDP                      16
## 6 DIE LINKE                15
## 7 <NA>                      6
## 8 fraktionslos              3
\end{verbatim}

Da \emph{NA} Fraktionen sind dabei die Reden von Ministern und Gästen.
Sie werden keiner Fraktion zugeordnet. Insgesamt gab es 144 Reden an
diesem Tag.

Uns interessieren natürlich nun nicht nur die Anzahl der Reden, sondern
auch deren Inhalt. Wir untersuchen hierfür alle Knoten eine Ebene unter
den ``rede''-Knoten.

\begin{Shaded}
\begin{Highlighting}[]
\NormalTok{prot_speeches <-}\StringTok{ }\NormalTok{prot_file }\OperatorTok
\StringTok{  }\KeywordTok{xml_find_all}\NormalTok{(}\StringTok{"//rede/*"}\NormalTok{)}
\end{Highlighting}
\end{Shaded}

Wir bauen wieder eine Funktion, um alle Inhalte der Reden zu
extrahieren. Diese Funktion ist ein wenig komplexer, da sie unter
anderem die Funktion \texttt{map()} aus dem \texttt{purrr} Packet nutzt
(ebenfalls \texttt{tidyverse}) -- für weitere Informationen über die
Funktion \texttt{map()} bietet sich dieses
\href{https://jennybc.github.io/purrr-tutorial/}{Tutorial an}.

Außerdem müssen die Rohdaten etwas angepasst werden, da die Aussagen des
Präsidiums sonst falsch zugeordnet werden.

\begin{Shaded}
\begin{Highlighting}[]
\NormalTok{get_speeches_df <-}\StringTok{ }\ControlFlowTok{function}\NormalTok{(x)\{}
\NormalTok{  raw <-}\StringTok{ }\NormalTok{x}
\NormalTok{  rede <-}\StringTok{ }\NormalTok{x }\OperatorTok\StringTok{ }\KeywordTok{xml_text}\NormalTok{()}
\NormalTok{  id <-}\StringTok{ }\NormalTok{x }\OperatorTok\StringTok{ }\KeywordTok{xml_node}\NormalTok{(}\StringTok{"redner"}\NormalTok{) }\OperatorTok\StringTok{ }\KeywordTok{xml_attr}\NormalTok{(}\StringTok{"id"}\NormalTok{)}
\NormalTok{  vorname <-}\StringTok{ }\NormalTok{x }\OperatorTok\StringTok{ }\KeywordTok{xml_node}\NormalTok{(}\StringTok{"vorname"}\NormalTok{) }\OperatorTok\StringTok{ }\KeywordTok{xml_text}\NormalTok{()}
\NormalTok{  nachname <-}\StringTok{ }\NormalTok{x }\OperatorTok\StringTok{ }\KeywordTok{xml_node}\NormalTok{(}\StringTok{"nachname"}\NormalTok{) }\OperatorTok\StringTok{ }\KeywordTok{xml_text}\NormalTok{()}
\NormalTok{  fraktion <-}\StringTok{ }\NormalTok{x }\OperatorTok\StringTok{ }\KeywordTok{xml_node}\NormalTok{(}\StringTok{"fraktion"}\NormalTok{) }\OperatorTok\StringTok{ }\KeywordTok{xml_text}\NormalTok{()}
\NormalTok{  rolle <-}\StringTok{ }\NormalTok{x }\OperatorTok\StringTok{ }\KeywordTok{xml_node}\NormalTok{(}\StringTok{"rolle_kurz"}\NormalTok{) }\OperatorTok\StringTok{ }\KeywordTok{xml_text}\NormalTok{()}
\NormalTok{  typ <-}\StringTok{ }\NormalTok{x }\OperatorTok\StringTok{ }\KeywordTok{xml_name}\NormalTok{()}
\NormalTok{  status <-}\StringTok{ }\NormalTok{x }\OperatorTok\StringTok{ }\KeywordTok{xml_attr}\NormalTok{(}\StringTok{"klasse"}\NormalTok{)}
  
  \KeywordTok{data_frame}\NormalTok{(raw, rede, id, vorname, nachname, fraktion, rolle, typ, status) }\OperatorTok
\StringTok{    }\KeywordTok{mutate}\NormalTok{(}\DataTypeTok{rede_id =} \KeywordTok{map}\NormalTok{(raw, }\OperatorTok{~}\KeywordTok{xml_parent}\NormalTok{(.) }\OperatorTok\StringTok{ }\KeywordTok{xml_attr}\NormalTok{(}\StringTok{"id"}\NormalTok{)) }\OperatorTok\StringTok{ }\KeywordTok{as.character}\NormalTok{()) }\OperatorTok
\StringTok{    }\KeywordTok{select}\NormalTok{(}\OperatorTok{-}\NormalTok{raw) }\OperatorTok
\StringTok{    }\KeywordTok{mutate}\NormalTok{(}\DataTypeTok{status =} \KeywordTok{ifelse}\NormalTok{(typ }\OperatorTok{==}\StringTok{ "kommentar"}\NormalTok{, typ, status)) }\OperatorTok
\StringTok{    }\KeywordTok{mutate}\NormalTok{(}\DataTypeTok{status =} \KeywordTok{ifelse}\NormalTok{(typ }\OperatorTok{==}\StringTok{ "name"}\NormalTok{, }\StringTok{"präsidium"}\NormalTok{, status)) }\OperatorTok
\StringTok{    }\KeywordTok{mutate}\NormalTok{(}\DataTypeTok{fraktion =} \KeywordTok{case_when}\NormalTok{(}
\NormalTok{      typ }\OperatorTok{==}\StringTok{ "name"}       \OperatorTok{~}\StringTok{ "präsidium"}\NormalTok{,}
      \OperatorTok{!}\KeywordTok{is.na}\NormalTok{(rolle)       }\OperatorTok{~}\StringTok{ "andere"}\NormalTok{,}
      \OtherTok{TRUE}                \OperatorTok{~}\StringTok{ }\NormalTok{fraktion)) }\OperatorTok
\StringTok{    }\KeywordTok{fill}\NormalTok{(id, vorname, nachname, fraktion) }\OperatorTok
\StringTok{    }\KeywordTok{mutate}\NormalTok{(prä}\DataTypeTok{sidium =} \KeywordTok{ifelse}\NormalTok{(fraktion }\OperatorTok{==}\StringTok{ "präsidium"}\NormalTok{, }\OtherTok{TRUE}\NormalTok{, }\OtherTok{FALSE}\NormalTok{)) }\OperatorTok
\StringTok{    }\KeywordTok{mutate}\NormalTok{(}\DataTypeTok{fraktion =} \KeywordTok{ifelse}\NormalTok{(fraktion }\OperatorTok{==}\StringTok{ "präsidium"}\NormalTok{, }\OtherTok{NA}\NormalTok{, fraktion)) }\OperatorTok
\StringTok{    }\KeywordTok{filter}\NormalTok{(}\OperatorTok{!}\NormalTok{status }\OperatorTok\StringTok{ }\KeywordTok{c}\NormalTok{(}\StringTok{"T_NaS"}\NormalTok{, }\StringTok{"T_Beratung"}\NormalTok{, }\StringTok{"T_fett"}\NormalTok{, }\StringTok{"redner"}\NormalTok{)) }\OperatorTok
\StringTok{    }\KeywordTok{filter}\NormalTok{(}\OperatorTok{!}\NormalTok{typ }\OperatorTok\StringTok{ }\KeywordTok{c}\NormalTok{(}\StringTok{"a"}\NormalTok{, }\StringTok{"fussnote"}\NormalTok{, }\StringTok{"sup"}\NormalTok{)) }\OperatorTok
\StringTok{    }\KeywordTok{select}\NormalTok{(rede_id, rede, id, vorname, nachname, fraktion, präsidium, typ, status)}
\NormalTok{\}}

\NormalTok{get_overview_df <-}\StringTok{ }\ControlFlowTok{function}\NormalTok{(x)\{}
\NormalTok{  rede_id <-}\StringTok{ }\NormalTok{x }\OperatorTok\StringTok{ }\KeywordTok{xml_attr}\NormalTok{(}\StringTok{"id"}\NormalTok{)}
\NormalTok{  redner_id <-}\StringTok{ }\NormalTok{x }\OperatorTok\StringTok{ }\KeywordTok{xml_node}\NormalTok{(}\StringTok{"redner"}\NormalTok{) }\OperatorTok\StringTok{ }\KeywordTok{xml_attr}\NormalTok{(}\StringTok{"id"}\NormalTok{)}
\NormalTok{  redner_vorname <-}\StringTok{ }\NormalTok{x }\OperatorTok\StringTok{ }\KeywordTok{xml_node}\NormalTok{(}\StringTok{"redner"}\NormalTok{) }\OperatorTok\StringTok{ }\KeywordTok{xml_node}\NormalTok{(}\StringTok{"vorname"}\NormalTok{) }\OperatorTok\StringTok{ }\KeywordTok{xml_text}\NormalTok{()}
\NormalTok{  redner_nachname <-}\StringTok{ }\NormalTok{x }\OperatorTok\StringTok{ }\KeywordTok{xml_node}\NormalTok{(}\StringTok{"redner"}\NormalTok{) }\OperatorTok\StringTok{ }\KeywordTok{xml_node}\NormalTok{(}\StringTok{"nachname"}\NormalTok{) }\OperatorTok\StringTok{ }\KeywordTok{xml_text}\NormalTok{()}
\NormalTok{  redner_fraktion <-}\StringTok{ }\NormalTok{x }\OperatorTok\StringTok{ }\KeywordTok{xml_node}\NormalTok{(}\StringTok{"redner"}\NormalTok{) }\OperatorTok\StringTok{ }\KeywordTok{xml_node}\NormalTok{(}\StringTok{"fraktion"}\NormalTok{) }\OperatorTok\StringTok{ }\KeywordTok{xml_text}\NormalTok{()}
\NormalTok{  redner_rolle <-}\StringTok{ }\NormalTok{x }\OperatorTok\StringTok{ }\KeywordTok{xml_node}\NormalTok{(}\StringTok{"rolle_kurz"}\NormalTok{) }\OperatorTok\StringTok{ }\KeywordTok{xml_text}\NormalTok{()}
\NormalTok{  sitzung <-}\StringTok{ }\NormalTok{x }\OperatorTok\StringTok{ }\KeywordTok{xml_find_first}\NormalTok{(}\StringTok{"//sitzungsnr"}\NormalTok{) }\OperatorTok\StringTok{ }\KeywordTok{xml_text}\NormalTok{() }\OperatorTok\StringTok{ }\KeywordTok{as.integer}\NormalTok{()}
\NormalTok{  datum <-}\StringTok{ }\NormalTok{x }\OperatorTok\StringTok{ }\KeywordTok{xml_find_first}\NormalTok{(}\StringTok{"//datum"}\NormalTok{) }\OperatorTok\StringTok{ }\KeywordTok{xml_attr}\NormalTok{(}\StringTok{"date"}\NormalTok{) }\OperatorTok\StringTok{ }\NormalTok{lubridate}\OperatorTok{::}\KeywordTok{dmy}\NormalTok{()}
\NormalTok{  wahlperiode <-}\StringTok{ }\NormalTok{x }\OperatorTok\StringTok{ }\KeywordTok{xml_find_first}\NormalTok{(}\StringTok{"//wahlperiode"}\NormalTok{) }\OperatorTok\StringTok{ }\KeywordTok{xml_text}\NormalTok{() }\OperatorTok\StringTok{ }\KeywordTok{as.integer}\NormalTok{()}
  
  \KeywordTok{data_frame}\NormalTok{(rede_id, redner_id, redner_vorname, redner_nachname, redner_fraktion, redner_rolle, sitzung, datum, wahlperiode)}
\NormalTok{\}}

\NormalTok{speeches_df <-}\StringTok{ }\KeywordTok{get_speeches_df}\NormalTok{(prot_speeches)}
\end{Highlighting}
\end{Shaded}

Mittels dieses Datenframes ist es nun möglich, nur die Aussagen von
beispielsweise Andrea Nahles zu untersuchen -- ohne Unterbrechungen und
Fragen von anderen Abgeordneten ``mitzuschneiden'' oder Aussagen des
Präsidiums mitzunehmen.

Hier ein Beispiel:

\begin{Shaded}
\begin{Highlighting}[]
\NormalTok{speeches_df }\OperatorTok
\StringTok{  }\KeywordTok{filter}\NormalTok{(typ }\OperatorTok{!=}\StringTok{ "kommentar"}\NormalTok{) }\OperatorTok
\StringTok{  }\KeywordTok{filter}\NormalTok{(präsidium }\OperatorTok{==}\StringTok{ }\OtherTok{FALSE}\NormalTok{) }\OperatorTok
\StringTok{  }\KeywordTok{filter}\NormalTok{(id }\OperatorTok{==}\StringTok{ "11003196"}\NormalTok{) }\OperatorTok
\StringTok{  }\KeywordTok{pull}\NormalTok{(rede) }\OperatorTok\StringTok{ }
\StringTok{  }\KeywordTok{cat}\NormalTok{(}\DataTypeTok{fill =} \OtherTok{TRUE}\NormalTok{)}
\end{Highlighting}
\end{Shaded}

\begin{verbatim}
## Herr Präsident! Meine lieben Kolleginnen und Kollegen! Auch in dieser Woche verabschieden SPD und CDU/CSU konkrete Gesetze, die für mehr Gerechtigkeit und für mehr Zusammenhalt in Deutschland sorgen werden. Wir erhöhen das Kindergeld um 10 Euro. Wir verabschieden eine Pflegereform, auf deren Grundlage deutlich mehr Pflegekräfte eingestellt werden und die Pflegekräfte insgesamt besser bezahlt werden. Wir schaffen Arbeitsplätze für Mitbürgerinnen und Mitbürger, die schon sehr lange arbeitslos sind. Und wir sichern die Rente auf dem jetzigen Niveau. Diese Regierung liefert. 
## Mit der heutigen Rentenreform vollziehen wir einen grundsätzlichen Richtungswechsel. Die alte Rentenformel sah vor, dass die Rente geringer steigt als die Löhne. Die neue Rentenformel stellt sicher: Die Renten steigen wie die Löhne. 
## Wir sichern damit ein Rentenniveau auf dem heutigen Level. Das ist wirklich eine sehr entscheidende Weichenstellung. Zusätzliche Vorsorge über Betriebsrenten oder privat ist eine gute Sache – wenn sie eben ergänzend gedacht ist, nicht ersetzend. Das ist der entscheidende Punkt für uns. 
## Denn die gesetzliche Rentenversicherung ist und bleibt die zentrale Säule im deutschen Rentensystem. 
## Die Rentenreform folgt einem einfachen Prinzip: Wer ein Leben lang arbeitet, der verdient auch einen anständigen Lebensabend, der verdient eine Rente, von der er auch leben kann. 
## Ich betone: Ich benutze den Begriff „verdient“ bewusst. Denn die Rente ist kein Almosen, und sie ist auch kein Luxus. Die Rente ist der gesellschaftliche Lohn für ein Leben voller Arbeit. Für die Mehrheit der Bevölkerung ist übrigens die gesetzliche Rente ihr größtes Vermögen. Sie bleibt die sicherste Form der Altersversorgung. 
## Uns ist die Stärkung der umlagefinanzierten Rente ja auch deswegen so wichtig, weil die Systeme, die vor allem auf private Absicherung ausgerichtet waren, letztendlich alle in der Finanzkrise deutlich gestrauchelt sind. Das ist ganz eindeutig der Fall gewesen. 
## Im Gegensatz zu den privaten steht die gesetzliche Rente blendend da. Würde man aus Beiträgen und Rentenansprüchen in der gesetzlichen Rente die Rendite berechnen, ergäbe sich ein stabiler Ertrag von 2 bis 3 Prozent pro Jahr, verlässlich und frei von Schwankungen. Das ist auf dem Kapitalmarkt momentan nicht zu kriegen, um es sehr deutlich zu sagen. 
## Die umlagefinanzierte Rente ist deswegen der kapitalgedeckten überlegen. 
## Ich spreche jetzt in diesem Hohen Haus auch etwas aus, was vielleicht nicht alle gerne hören: Entweder wir sichern heute das Rentenniveau auf dem jetzigen Stand bis zum Jahr 2025 und nach dem Willen der SPD auch weiter darüber hinaus, 
## oder wir lassen zu, dass die Renten immer weiter sinken und entwertet werden. 
## Wenn wir das aber zulassen, muss die junge Generation einem solchen System irgendwann das Vertrauen entziehen. Denn warum sollte ausgerechnet die junge Generation jahrzehntelang Beiträge zahlen, wenn sie am Ende keine Sicherheit darüber hat, was sie rausbekommt? Das ist doch Unsinn. 
## Deswegen ist aus meiner Sicht die Sicherung des Rentenniveaus in diesem System auch wichtig im Sinne der Generationengerechtigkeit. Ein garantiertes Rentenniveau schafft für die junge Generation nämlich die Sicherheit, dass sie sich eben am Ende auch auf dieses System der gesetzlichen Rentenversicherung verlassen kann. 
## Jetzt sagen manche, das sei nicht finanzierbar. Das ist ein ziemlich scheinheiliges Argument. 
## Denn niemand wird ja wohl bestreiten, dass das Geld für eine auskömmliche Rente im Jahre, sagen wir, 2040 auch immer irgendwo herkommen muss. Die einzige Frage ist doch: Was ist der beste und der gerechteste Weg, dies dann zu finanzieren? Soll die heutige Arbeitnehmergeneration sowohl die Renten von heute finanzieren und gleichzeitig privat noch die eigene Rente aufstocken? Damit würden viele Arbeitnehmerinnen und Arbeitnehmer komplett überfordert. Oder soll auch die heutige Arbeitnehmergeneration sich darauf verlassen können, dass auch sie im Alter eine von ihren Kindern und dann auch durch zusätzliche Steuermittel finanzierte Rente bekommt? Die Frage ist doch nicht, ob, sondern die Frage ist, wie wir die Renten und die Garantie eines Rentenniveaus in Zukunft finanzieren. Darüber lohnt sich jeder Streit; gar keine Frage. 
## Das, was wir heute beschließen, ist finanziert. Bis 2025 ist das Rentenniveau klar gesichert. 
## Wir steigen darüber hinaus in die Bildung einer Demografierücklage ein. 
## Damit schaffen wir die Voraussetzung, um den Steueranteil zur Finanzierung der Rentenversicherung systematisch auf- und ausbauen zu können. 
## Das wird wahrscheinlich auch der Weg der Zukunft sein. Darüber wird aber in der Rentenkommission noch weiter diskutiert werden. 
## Wenn es aber etwas gibt, was wir klären müssen, dann ist das doch die Frage: Wollen wir in Zukunft wirklich auf die gesetzliche Rentenversicherung als wesentliche Säule unseres Rentensystems setzen, ja oder nein? 
## Einen Weg zur Finanzierung werden wir in einem reichen Land wie Deutschland sicherlich finden, und zwar einen gerechten, wenn es nach der SPD geht. 
## Letzter Satz. Wenn es also einen Gradmesser für die soziale Sicherheit in Deutschland gibt, dann ist das aus meiner Sicht eine gute Alterssicherung. Für eine gute Altersversorgung sorgen wir mit diesem Rentenpaket heute und jetzt. 
## Vielen Dank.
\end{verbatim}

Somit könnten wir für dieses Protokoll die einzelnen Reden (aber zum
Beispiel auch Zwischenfragen) von Abgeordneten gezielt auf deren Inhalte
untersuchen. Wir können noch nicht die Zwischenrufe und den Applaus nach
Abgeordneten bzw. Fraktionen auswerten. Dies wäre mit sogenannten
\emph{regular experesions} aber möglich.

Wie wir alle aktuellen Protokolle auswerten, behandeln wir in Daten wir
in Kapitel \ref{data-02}. Die beiden Funktionen speichern wir im Ordner
``functions''.

\hypertarget{data-02}{%
\chapter{Erhebung der Daten des 19. Bundestags}\label{data-02}}

Leider gibt es keine Möglichkeit, die XML-Protokolle des Deutschen
Bundestages gesammelt herunterzuladen. Zwar finden sich auf \emph{Open
Data}-Seite des Bundestags\footnote{\url{https://www.bundestag.de/service/opendata}}
die Verweise auf die Bundestagsprotkolle, allerdings können diese nur
umständlich einezln heruntergeladen werden. Durch das Auslesen der
Netzwerkdaten nach einem Klick auf die nächsten fünf Protkolle eine
Webseite gefunden werden, auf welcher jeweils fünf Protkolle gespeichert
sind\footnote{\url{https://www.bundestag.de/ajax/filterlist/de/service/opendata/-/543410?offset=0}}.
Durch setzen des \texttt{offset=0} auf
\texttt{5,\ 10,\ 15,\ 20,\ \ldots{}} können wir die weiteren Protokolle
abrufen.

Mittels \texttt{rvest} und \texttt{xml2} ziehen wir uns zunächst die
Nummer des letzten Bundestagsprotokolls. Über die Funktion
\texttt{seq()} und \texttt{paste0()} können wir anschließend alle für
den weiteren Verlauf notwendigen URLs erstellen.

\begin{Shaded}
\begin{Highlighting}[]
\NormalTok{bt_website <-}\StringTok{ "https://www.bundestag.de/ajax/filterlist/de/service/opendata/-/543410"}

\NormalTok{last_protocol <-}\StringTok{ }\NormalTok{bt_website }\OperatorTok\StringTok{ }
\StringTok{  }\KeywordTok{read_html}\NormalTok{() }\OperatorTok
\StringTok{  }\KeywordTok{xml_find_first}\NormalTok{(}\StringTok{"//strong"}\NormalTok{) }\OperatorTok\StringTok{ }
\StringTok{  }\KeywordTok{xml_text}\NormalTok{(}\DataTypeTok{trim =} \OtherTok{TRUE}\NormalTok{) }\OperatorTok
\StringTok{  }\KeywordTok{str_extract}\NormalTok{(}\StringTok{"}\CharTok{\textbackslash{}\textbackslash{}}\StringTok{d+"}\NormalTok{)}

\NormalTok{prot_websites <-}\StringTok{ }\KeywordTok{paste0}\NormalTok{(bt_website, }\StringTok{"?offset="}\NormalTok{, }\KeywordTok{seq}\NormalTok{(}\DecValTok{0}\NormalTok{, last_protocol, }\DecValTok{5}\NormalTok{))}
\end{Highlighting}
\end{Shaded}

Insgesamt müssen wir also 13 Webseiten aufrufen und uns jeweils fünf
Protokolle herunterladen.

Der Aufwand ist hier noch überschaubar. Natürlich laden wir die
Protokolle dennoch nicht per Hand herunter, sondern erstellen uns
hierfür ein kleines Script. Da die meisten Rechner auch mit mehr als nur
einem Prozessor ausgestattet sind, können wir die Funktion auch auf
mehreren Prozessoren ausführen. Wir sprechen hier von
\emph{multiprocessing}. Dies realisieren wir über das Packet
\texttt{furrr} (Vaughan/Dancho
\protect\hyperlink{ref-vaughan_2018}{2018}) - eine Abwandlung des
bereits genutzten \texttt{purrr}.

Wir schreiben uns zunächst eine Funktion, um die Links der Webseiten zu
extrahieren.

\begin{Shaded}
\begin{Highlighting}[]
\NormalTok{get_prot_links <-}\StringTok{ }\ControlFlowTok{function}\NormalTok{(x)\{}
\NormalTok{  x }\OperatorTok
\StringTok{    }\KeywordTok{read_html}\NormalTok{() }\OperatorTok
\StringTok{    }\KeywordTok{html_nodes}\NormalTok{(}\StringTok{".bt-link-dokument"}\NormalTok{) }\OperatorTok
\StringTok{    }\KeywordTok{html_attr}\NormalTok{(}\StringTok{"href"}\NormalTok{) }\OperatorTok
\StringTok{    }\KeywordTok{paste0}\NormalTok{(}\StringTok{"https://www.bundestag.de"}\NormalTok{, .)}
\NormalTok{  \}}

\KeywordTok{get_prot_links}\NormalTok{(bt_website)}
\end{Highlighting}
\end{Shaded}

\begin{verbatim}
## [1] "https://www.bundestag.de/blob/578466/7430bccaf792e7bc55e84d5e64675820/19062-data.xml"
## [2] "https://www.bundestag.de/blob/577958/e2063c0f51a32690a269f48aa6102c1d/19061-data.xml"
## [3] "https://www.bundestag.de/blob/577622/da97888b713abb16ed2070836504b83a/19060-data.xml"
## [4] "https://www.bundestag.de/blob/575138/b5395a975d1c55838da0e52251018160/19059-data.xml"
## [5] "https://www.bundestag.de/blob/574826/0e3659e11c1c3cdbfa621369cd16735a/19058-data.xml"
\end{verbatim}

Dies wenden wir nun auf alle 13 Webseiten an und speichern die Dateien
anschließend im Ordner \texttt{data\textbackslash{}protokolle}.

\begin{Shaded}
\begin{Highlighting}[]
\KeywordTok{library}\NormalTok{(furrr)}
\KeywordTok{plan}\NormalTok{(multiprocess)}

\NormalTok{prot_links <-}\StringTok{ }\KeywordTok{future_map}\NormalTok{(prot_websites, }\OperatorTok{~}\KeywordTok{get_prot_links}\NormalTok{(.)) }\OperatorTok\StringTok{ }\KeywordTok{unlist}\NormalTok{()}

\NormalTok{prot_links }\OperatorTok\StringTok{ }\KeywordTok{future_map}\NormalTok{(}\OperatorTok{~}\KeywordTok{download.file}\NormalTok{(., }\KeywordTok{file.path}\NormalTok{(}\StringTok{"data/protokolle"}\NormalTok{, }\KeywordTok{basename}\NormalTok{(.))))}
\end{Highlighting}
\end{Shaded}

Wir waren erfolgreich und konnten in wenigsten Sekunden alle aktuellen
Protokolle herunterladen. Wir können sie jetzt auslesen und dabei auch
unsere bereits erstellten Funktionen verwenden.

\begin{Shaded}
\begin{Highlighting}[]
\NormalTok{prot_files <-}\StringTok{ }\KeywordTok{list.files}\NormalTok{(}\StringTok{"data/protokolle"}\NormalTok{, }\DataTypeTok{full.names =} \OtherTok{TRUE}\NormalTok{)}

\NormalTok{prot_extract <-}\StringTok{ }\KeywordTok{map}\NormalTok{(prot_files, }\OperatorTok{~}\KeywordTok{read_html}\NormalTok{(.) }\OperatorTok\StringTok{ }\KeywordTok{xml_find_all}\NormalTok{(}\StringTok{"//rede"}\NormalTok{))}

\KeywordTok{class}\NormalTok{(prot_extract) <-}\StringTok{ "xml_nodeset"}

\NormalTok{prot_overview <-}\StringTok{ }\KeywordTok{map_dfr}\NormalTok{(prot_extract, }\OperatorTok{~}\KeywordTok{get_overview_df}\NormalTok{(.))}
\end{Highlighting}
\end{Shaded}

\hypertarget{uberblick-der-reden}{%
\section{Überblick der Reden}\label{uberblick-der-reden}}

Wir konnten nun alle Dateien herunterladen und anschließend alle
Protkolle in R einlesen. Mit unserer Funktion
\texttt{get\_overview\_df()} konnten wir alle Protokolle in einen für
uns passenden Datenframe umwandeln. Insgesamt können wir derzeit 5.777
Reden des aktuellen Bundestags untersuchen -- etwa nach der Person,
welche die meisten reden gehalten hat.

\begin{Shaded}
\begin{Highlighting}[]
\NormalTok{prot_overview }\OperatorTok\StringTok{ }
\StringTok{  }\KeywordTok{group_by}\NormalTok{(redner_id, redner_vorname, redner_nachname,}
\NormalTok{           redner_fraktion, redner_rolle) }\OperatorTok\StringTok{ }
\StringTok{  }\KeywordTok{summarise}\NormalTok{(}\DataTypeTok{reden =} \KeywordTok{n}\NormalTok{()) }\OperatorTok
\StringTok{  }\KeywordTok{arrange}\NormalTok{(}\OperatorTok{-}\NormalTok{reden)}
\end{Highlighting}
\end{Shaded}

\begin{verbatim}
## # A tibble: 773 x 6
## # Groups:   redner_id, redner_vorname, redner_nachname, redner_fraktion
## #   [728]
##    redner_id redner_vorname redner_nachname redner_fraktion redner_rolle
##    <chr>     <chr>          <chr>           <chr>           <chr>       
##  1 11002617  Peter          Altmaier        <NA>            Bundesminis~
##  2 999990073 Olaf           Scholz          <NA>            Bundesminis~
##  3 11004427  Volker         Ullrich         CDU/CSU         <NA>        
##  4 11001478  Angela         Merkel          <NA>            Bundeskanzl~
##  5 11004809  Heiko          Maas            <NA>            Bundesminis~
##  6 11004851  Frauke         Petry           fraktionslos    <NA>        
##  7 11004798  Alexander Graf Lambsdorff      FDP             <NA>        
##  8 11003625  Andreas        Scheuer         <NA>            Bundesminis~
##  9 999990074 Svenja         Schulze         <NA>            Bundesminis~
## 10 11003638  Jens           Spahn           <NA>            Bundesminis~
## # ... with 763 more rows, and 1 more variable: reden <int>
\end{verbatim}

Die Minister\_innen haben am häuiigsten im Bundestag geredet, der MdB
Volker Ullrich landet erst af Platz 4 mit insgesamt vierzig Reden in der
aktuellen Wahlperiode. Überraschenderweise hat Frauke Petry sehr viele
Reden gehalten: 35 Stück. Das sind deutlich mehr als in ihrer damaligen
Zeit im Landesparlament.

Wir können die Reden auch nach Fraktionen auswerten:

\begin{Shaded}
\begin{Highlighting}[]
\NormalTok{prot_overview }\OperatorTok
\StringTok{  }\KeywordTok{group_by}\NormalTok{(redner_fraktion) }\OperatorTok
\StringTok{  }\KeywordTok{summarise}\NormalTok{(}\DataTypeTok{reden =} \KeywordTok{n}\NormalTok{()) }\OperatorTok
\StringTok{  }\KeywordTok{arrange}\NormalTok{(}\OperatorTok{-}\NormalTok{reden) }\OperatorTok
\StringTok{  }\NormalTok{knitr}\OperatorTok{::}\KeywordTok{kable}\NormalTok{(}\DataTypeTok{caption =} \StringTok{"Reden Nach Fraktion im Deutschen Bundestag"}\NormalTok{, }\DataTypeTok{booktabs =} \OtherTok{TRUE}\NormalTok{)}
\end{Highlighting}
\end{Shaded}

\begin{table}

\caption{\label{tab:unnamed-chunk-17}Reden Nach Fraktion im Deutschen Bundestag}
\centering
\begin{tabular}[t]{lr}
\toprule
redner\_fraktion & reden\\
\midrule
CDU/CSU & 1353\\
SPD & 962\\
AfD & 784\\
FDP & 683\\
BÜNDNIS 90/DIE GRÜNEN & 662\\
\addlinespace
NA & 662\\
DIE LINKE & 606\\
fraktionslos & 63\\
Bremen & 1\\
Bündnis 90/Die Grünen & 1\\
\bottomrule
\end{tabular}
\end{table}

Mit auswertung der Daten nach Fraktion stellen wir kleinere Probleme
Fest: Anscheinend wurden die Grüne in einem Dokument nicht in der
üblichen Schreibweise geschrieben. Und die Fraktion ``Bremen'' ist
vermutlich auch in der falschen Spalte gelandet - es handelt sich
nämlich um eine Rede des Bremer Bürgermeisters. Beide ändern wir
entsprechend:

\begin{Shaded}
\begin{Highlighting}[]
\NormalTok{prot_overview <-}\StringTok{ }\NormalTok{prot_overview }\OperatorTok
\StringTok{  }\KeywordTok{mutate}\NormalTok{(}\DataTypeTok{redner_fraktion =} \KeywordTok{ifelse}\NormalTok{(redner_fraktion }\OperatorTok{==}\StringTok{ "Bündnis 90/Die Grünen", "}\NormalTok{BÜNDNIS }\DecValTok{90}\OperatorTok{/}\NormalTok{DIE GRÜNEN}\StringTok{", redner_fraktion)) %>%}
\StringTok{  mutate(redner_fraktion = ifelse(redner_fraktion == "}\NormalTok{Bremen}\StringTok{", NA, redner_fraktion))}
\end{Highlighting}
\end{Shaded}

\hypertarget{inhalte-der-reden}{%
\section{Inhalte der Reden}\label{inhalte-der-reden}}

WIRD NOCH ERGÄNZT

\begin{Shaded}
\begin{Highlighting}[]
\NormalTok{speech_extract <-}\StringTok{ }\KeywordTok{map}\NormalTok{(prot_files, }\OperatorTok{~}\KeywordTok{read_html}\NormalTok{(.) }\OperatorTok\StringTok{ }\KeywordTok{xml_find_all}\NormalTok{(}\StringTok{"//rede/*"}\NormalTok{))}

\KeywordTok{class}\NormalTok{(speech_extract) <-}\StringTok{ "xml_nodeset"}

\NormalTok{prot_speeches <-}\StringTok{ }\KeywordTok{map_dfr}\NormalTok{(speech_extract, }\OperatorTok{~}\KeywordTok{get_speeches_df}\NormalTok{(.))}
\end{Highlighting}
\end{Shaded}

\hypertarget{daten-der-abgeordneten}{%
\section{Daten der Abgeordneten}\label{daten-der-abgeordneten}}

Wir haben nun die Reden als auch die Inhalte der Reden im Bundestag
heruntergeladen. Dabei haben wir auch schon Verweise auf die Fraktionen
und auch die Rollen (beispielsweise die Rede als Bundesminister\_in)
generieren können. Es fehlen uns jedoch noch Aussagen über das
Geschlecht der Redner\_innen (und weitere Daten -- etwa Alter,
Ausschussmitgliedschaften und Amtszeit).

Diese können wir abermals über das Open Data Portal\footnote{\url{https://www.bundestag.de/service/opendata}}
des Bundestags herunterladen. Es handelt sich um eine
\texttt{ZIP}-Datei, welche wir zunächst herunterladen und anschließed
extrahieren.

\begin{Shaded}
\begin{Highlighting}[]
\NormalTok{link_zip <-}\StringTok{ "https://www.bundestag.de/blob/472878/e207ab4b38c93187c6580fc186a95f38/mdb-stammdaten-data.zip"}

\KeywordTok{download.file}\NormalTok{(link_zip, }\KeywordTok{file.path}\NormalTok{(}\StringTok{"data/"}\NormalTok{, }\KeywordTok{basename}\NormalTok{(link_zip)))}

\KeywordTok{unzip}\NormalTok{(}\StringTok{"data/mdb-stammdaten-data.zip"}\NormalTok{, }\DataTypeTok{exdir =} \StringTok{"data/"}\NormalTok{)}
\end{Highlighting}
\end{Shaded}

\begin{Shaded}
\begin{Highlighting}[]
\NormalTok{file_stammdaten <-}\StringTok{ }\KeywordTok{read_xml}\NormalTok{(}\StringTok{"data/MDB_STAMMDATEN.XML"}\NormalTok{)}
\end{Highlighting}
\end{Shaded}

Nachdem die Datei heruntergeladen und ausgelesen wurde, können wir die
Daten wieder einen \emph{Dataframe} umwandeln.

\begin{Shaded}
\begin{Highlighting}[]
\NormalTok{get_data_mdb <-}\StringTok{ }\ControlFlowTok{function}\NormalTok{(x)\{}
\NormalTok{  id <-}\StringTok{ }\NormalTok{x }\OperatorTok\StringTok{ }\KeywordTok{xml_find_all}\NormalTok{(}\StringTok{"//ID"}\NormalTok{) }\OperatorTok\StringTok{ }\KeywordTok{xml_text}\NormalTok{()}
\NormalTok{  geschlecht <-}\StringTok{ }\NormalTok{x }\OperatorTok\StringTok{ }\KeywordTok{xml_find_all}\NormalTok{(}\StringTok{"//ID/following-sibling::BIOGRAFISCHE_ANGABEN/GESCHLECHT"}\NormalTok{) }\OperatorTok\StringTok{ }\KeywordTok{xml_text}\NormalTok{()}
\NormalTok{  geburtsjahr <-}\StringTok{ }\NormalTok{x }\OperatorTok\StringTok{ }\KeywordTok{xml_find_all}\NormalTok{(}\StringTok{"//ID/following-sibling::BIOGRAFISCHE_ANGABEN/GEBURTSDATUM"}\NormalTok{) }\OperatorTok\StringTok{ }\KeywordTok{xml_text}\NormalTok{() }\OperatorTok\StringTok{ }\KeywordTok{as.integer}\NormalTok{()}
\NormalTok{  partei <-}\StringTok{ }\NormalTok{x }\OperatorTok\StringTok{ }\KeywordTok{xml_find_all}\NormalTok{(}\StringTok{"//ID/following-sibling::BIOGRAFISCHE_ANGABEN/PARTEI_KURZ"}\NormalTok{) }\OperatorTok\StringTok{ }\KeywordTok{xml_text}\NormalTok{()}
\NormalTok{  wahlperioden <-}\StringTok{ }\NormalTok{x }\OperatorTok\StringTok{ }\KeywordTok{xml_find_all}\NormalTok{(}\StringTok{"//ID/following-sibling::WAHLPERIODEN"}\NormalTok{)}
  
  \KeywordTok{data_frame}\NormalTok{(id, geschlecht, geburtsjahr, partei, wahlperioden) }\OperatorTok
\StringTok{    }\KeywordTok{mutate}\NormalTok{(}\DataTypeTok{wahlperioden =} \KeywordTok{map}\NormalTok{(wahlperioden, }\OperatorTok{~}\KeywordTok{xml_nodes}\NormalTok{(., }\StringTok{"WP"}\NormalTok{) }\OperatorTok\StringTok{ }\KeywordTok{xml_text}\NormalTok{() }\OperatorTok\StringTok{ }\KeywordTok{as.integer}\NormalTok{())) }\OperatorTok
\StringTok{    }\KeywordTok{mutate}\NormalTok{(}\DataTypeTok{anzahl_wahlperioden =} \KeywordTok{map}\NormalTok{(wahlperioden, }\OperatorTok{~}\KeywordTok{length}\NormalTok{(.)) }\OperatorTok\StringTok{ }\KeywordTok{unlist}\NormalTok{())}
\NormalTok{\}}

\NormalTok{data_mdb <-}\StringTok{ }\KeywordTok{get_data_mdb}\NormalTok{(file_stammdaten)}

\NormalTok{data_mdb}
\end{Highlighting}
\end{Shaded}

\begin{verbatim}
## # A tibble: 4,073 x 6
##    id       geschlecht geburtsjahr partei wahlperioden anzahl_wahlperioden
##    <chr>    <chr>            <int> <chr>  <list>                     <int>
##  1 11000001 männlich          1930 CDU    <int [7]>                      7
##  2 11000002 männlich          1909 FDP    <int [5]>                      5
##  3 11000003 weiblich          1913 CDU    <int [3]>                      3
##  4 11000004 weiblich          1933 CDU    <int [2]>                      2
##  5 11000005 männlich          1950 CDU    <int [5]>                      5
##  6 11000007 männlich          1919 SPD    <int [4]>                      4
##  7 11000008 männlich          1912 CDU    <int [1]>                      1
##  8 11000009 männlich          1876 CDU    <int [5]>                      5
##  9 11000010 weiblich          1944 SPD    <int [4]>                      4
## 10 11000011 männlich          1920 CDU    <int [3]>                      3
## # ... with 4,063 more rows
\end{verbatim}

Die Namen brauchen wir nicht zwingend, da wir dies auch den bestehenden
Daten entnehmen können (außerdem haben wir ein kleine Problem mit
Abgeordneten, welche während oder zwischen den Wahlperioden den
Nachnamen ändern oder ergänzen). Wichtig ist hier die ID -- mit dieser
können wir die einzelnen Reden den Abgeordneten und schließich auch dem
jeweiligen Geschlecht zuordnen.

Bei den Stammdaten handelt es sich um die Daten aller
Bundestagsabgeordnetem seit dem 1. Bundestag, welcher 14. August 1949
gewählt wurde. Insgesamt waren 4.073 Personen im Bundestag vertreten.
Wie viele davon weiblich und männlich waren, lässt sich wenigen Zeilen
Code darstellen. Die letzten beiden Zeilen des Codes beziehen sich nur
auf eine bessere Darstellung in der PDF und der Webseite.

\begin{Shaded}
\begin{Highlighting}[]
\NormalTok{data_mdb }\OperatorTok\StringTok{ }
\StringTok{  }\KeywordTok{group_by}\NormalTok{(geschlecht) }\OperatorTok\StringTok{ }
\StringTok{  }\KeywordTok{summarise}\NormalTok{(}\DataTypeTok{n =} \KeywordTok{n}\NormalTok{()) }\OperatorTok
\StringTok{  }\KeywordTok{mutate}\NormalTok{(}\DataTypeTok{freq =}\NormalTok{ n }\OperatorTok{/}\StringTok{ }\KeywordTok{sum}\NormalTok{(n)) }\OperatorTok
\StringTok{  }\KeywordTok{arrange}\NormalTok{(}\OperatorTok{-}\NormalTok{n) }\OperatorTok
\StringTok{  }\NormalTok{knitr}\OperatorTok{::}\KeywordTok{kable}\NormalTok{(}\DataTypeTok{caption =} \StringTok{"Bundestagsabgeordnete aller Wahlperioden nach Geschlecht"}\NormalTok{, }\DataTypeTok{booktabs =} \OtherTok{TRUE}\NormalTok{)}
\end{Highlighting}
\end{Shaded}

\begin{table}

\caption{\label{tab:unnamed-chunk-23}Bundestagsabgeordnete aller Wahlperioden nach Geschlecht}
\centering
\begin{tabular}[t]{lrr}
\toprule
geschlecht & n & freq\\
\midrule
männlich & 3223 & 0.7913086\\
weiblich & 850 & 0.2086914\\
\bottomrule
\end{tabular}
\end{table}

Die große Mehrheit der im Bundestag vertretenen Person ist männlich. Nur
eine Minderheit von 20,9\% der Abgeordneten war oder ist weiblich. Ein
nicht-binäres Geschlecht oder eine Verweigerung der Angabe des
Geschlechts wurde bisher noch nicht angegeben oder nicht dokumentiert --
trans- und intersexuelle Menschen sind im Bundestag somit auch weiterhin
kaum vertreten.\footnote{Eine Ausnahme ist etwa der Politiker Christian
  Schenk, welcher sich nach seiner Zeit im Bundestag als Transmann
  outete und eine Geschlechtsumwandlung vollzog.}

Wir können mit diesen Daten auch mit wenig aufwand die Frauenquote in
den jeweiligen Wahlperioden darstellen.

\begin{Shaded}
\begin{Highlighting}[]
\NormalTok{data_mdb }\OperatorTok
\StringTok{  }\KeywordTok{unnest}\NormalTok{() }\OperatorTok
\StringTok{  }\KeywordTok{group_by}\NormalTok{(geschlecht, wahlperioden) }\OperatorTok
\StringTok{  }\KeywordTok{summarise}\NormalTok{(}\DataTypeTok{n =} \KeywordTok{n}\NormalTok{()) }\OperatorTok
\StringTok{  }\KeywordTok{group_by}\NormalTok{(wahlperioden) }\OperatorTok
\StringTok{  }\KeywordTok{mutate}\NormalTok{(}\DataTypeTok{freq =}\NormalTok{ n }\OperatorTok{/}\StringTok{ }\KeywordTok{sum}\NormalTok{(n)) }\OperatorTok
\StringTok{  }\KeywordTok{rename}\NormalTok{(}\DataTypeTok{Geschlecht =}\NormalTok{ geschlecht) }\OperatorTok
\StringTok{  }\KeywordTok{ggplot}\NormalTok{(}\KeywordTok{aes}\NormalTok{(}\DataTypeTok{x =}\NormalTok{ wahlperioden, }\DataTypeTok{y =}\NormalTok{ freq, }\DataTypeTok{colour =}\NormalTok{ Geschlecht)) }\OperatorTok{+}
\StringTok{  }\KeywordTok{geom_line}\NormalTok{() }\OperatorTok{+}
\StringTok{  }\KeywordTok{labs}\NormalTok{(}\DataTypeTok{title =} \StringTok{"Geschlechteranteile im Deutschen Bundestag"}\NormalTok{, }
       \DataTypeTok{subtitle =} \StringTok{"Auswertung des 1. bis 19. Deutschen Bundestages"}\NormalTok{,}
       \DataTypeTok{x =} \StringTok{"Wahlperiode"}\NormalTok{,}
       \DataTypeTok{y =} \StringTok{"Anteil in Prozent"}\NormalTok{) }\OperatorTok{+}
\StringTok{  }\KeywordTok{theme_minimal}\NormalTok{()}
\end{Highlighting}
\end{Shaded}

\begin{figure}

{\centering \includegraphics{bundestag_files/figure-latex/gender-wp-bt-1} 

}

\caption{Auswertung der Geschlechteranteile aller Deutscher Bundestage}\label{fig:gender-wp-bt}
\end{figure}

Wie in Abb. \ref{fig:gender-wp-bt} zu erkennen, ist mit der aktuellen
Wahlperiode der Anteil an Frauen im Bundestag deutlich zurückgegangen.
Die Anteil an Frauen im aktuellen Bundestag beträgt aktuell 30,7 \%.

\hypertarget{auswertung}{%
\chapter{Auswertung}\label{auswertung}}

Nachdem alle Daten abgerufen wurden, können wir uns endlich an die
Auswertung machen. Zunächst, wie hoch ist der Frauenanteil nach den
einzelnen Fraktionen?

\begin{Shaded}
\begin{Highlighting}[]
\NormalTok{data_mdb }\OperatorTok
\StringTok{  }\KeywordTok{unnest}\NormalTok{() }\OperatorTok\StringTok{ }
\StringTok{  }\KeywordTok{filter}\NormalTok{(wahlperioden }\OperatorTok{==}\StringTok{ }\DecValTok{19}\NormalTok{) }\OperatorTok
\StringTok{  }\KeywordTok{group_by}\NormalTok{(geschlecht, partei) }\OperatorTok
\StringTok{  }\KeywordTok{summarise}\NormalTok{(}\DataTypeTok{n =} \KeywordTok{n}\NormalTok{()) }\OperatorTok
\StringTok{  }\KeywordTok{group_by}\NormalTok{(partei) }\OperatorTok
\StringTok{  }\KeywordTok{mutate}\NormalTok{(}\DataTypeTok{freq =}\NormalTok{ n}\OperatorTok{/}\KeywordTok{sum}\NormalTok{(n)) }\OperatorTok
\StringTok{  }\KeywordTok{filter}\NormalTok{(geschlecht }\OperatorTok{==}\StringTok{ "weiblich"}\NormalTok{) }\OperatorTok
\StringTok{  }\KeywordTok{arrange}\NormalTok{(}\OperatorTok{-}\NormalTok{freq) }\OperatorTok
\StringTok{  }\KeywordTok{mutate}\NormalTok{(}\DataTypeTok{freq =}\NormalTok{ scales}\OperatorTok{::}\KeywordTok{percent}\NormalTok{(freq)) }\OperatorTok
\StringTok{  }\KeywordTok{select}\NormalTok{(}\OperatorTok{-}\NormalTok{geschlecht)}
\end{Highlighting}
\end{Shaded}

\begin{verbatim}
## # A tibble: 8 x 3
## # Groups:   partei [8]
##   partei                    n freq 
##   <chr>                 <int> <chr>
## 1 BÜNDNIS 90/DIE GRÜNEN    39 58.2%
## 2 DIE LINKE.               37 53.6%
## 3 Blaue                     1 50.0%
## 4 SPD                      65 41.9%
## 5 FDP                      19 23.7%
## 6 CDU                      41 20.5%
## 7 CSU                       8 17.4%
## 8 AfD                      10 10.9%
\end{verbatim}

Wie hoch ist der Anteil an Reden von Frauen im Bundestag?

\begin{Shaded}
\begin{Highlighting}[]
\NormalTok{data_mdb }\OperatorTok
\StringTok{  }\KeywordTok{unnest}\NormalTok{() }\OperatorTok
\StringTok{  }\KeywordTok{filter}\NormalTok{(wahlperioden }\OperatorTok{==}\StringTok{ }\DecValTok{19}\NormalTok{) }\OperatorTok
\StringTok{  }\KeywordTok{select}\NormalTok{(}\OperatorTok{-}\NormalTok{geburtsjahr, }\OperatorTok{-}\NormalTok{anzahl_wahlperioden, }\OperatorTok{-}\NormalTok{wahlperioden) }\OperatorTok
\StringTok{  }\KeywordTok{left_join}\NormalTok{(prot_overview, }\DataTypeTok{by =} \KeywordTok{c}\NormalTok{(}\StringTok{"id"}\NormalTok{ =}\StringTok{ "redner_id"}\NormalTok{)) }\OperatorTok
\StringTok{  }\KeywordTok{group_by}\NormalTok{(geschlecht) }\OperatorTok
\StringTok{  }\KeywordTok{summarise}\NormalTok{(}\DataTypeTok{n =} \KeywordTok{n}\NormalTok{()) }\OperatorTok
\StringTok{  }\KeywordTok{mutate}\NormalTok{(}\DataTypeTok{freq =}\NormalTok{ n}\OperatorTok{/}\KeywordTok{sum}\NormalTok{(n)) }\OperatorTok
\StringTok{  }\KeywordTok{mutate}\NormalTok{(}\DataTypeTok{freq =}\NormalTok{ scales}\OperatorTok{::}\KeywordTok{percent}\NormalTok{(freq))}
\end{Highlighting}
\end{Shaded}

\begin{verbatim}
## # A tibble: 2 x 3
##   geschlecht     n freq 
##   <chr>      <int> <chr>
## 1 männlich    3894 68.8%
## 2 weiblich    1765 31.2%
\end{verbatim}

Überraschenderweise höher als der Anteil an weiblichen Abgeordneten. Wie
sieht es für die einzelnen Parteien an. Untersuchen wir den Anteil an
Reden von Frauen.

\begin{Shaded}
\begin{Highlighting}[]
\NormalTok{data_mdb }\OperatorTok
\StringTok{  }\KeywordTok{unnest}\NormalTok{() }\OperatorTok
\StringTok{  }\KeywordTok{filter}\NormalTok{(wahlperioden }\OperatorTok{==}\StringTok{ }\DecValTok{19}\NormalTok{) }\OperatorTok
\StringTok{  }\KeywordTok{select}\NormalTok{(}\OperatorTok{-}\NormalTok{geburtsjahr, }\OperatorTok{-}\NormalTok{anzahl_wahlperioden, }\OperatorTok{-}\NormalTok{wahlperioden) }\OperatorTok
\StringTok{  }\KeywordTok{left_join}\NormalTok{(prot_overview, }\DataTypeTok{by =} \KeywordTok{c}\NormalTok{(}\StringTok{"id"}\NormalTok{ =}\StringTok{ "redner_id"}\NormalTok{)) }\OperatorTok
\StringTok{  }\KeywordTok{group_by}\NormalTok{(geschlecht, redner_fraktion) }\OperatorTok
\StringTok{  }\KeywordTok{summarise}\NormalTok{(}\DataTypeTok{n =} \KeywordTok{n}\NormalTok{()) }\OperatorTok
\StringTok{  }\KeywordTok{group_by}\NormalTok{(redner_fraktion) }\OperatorTok
\StringTok{  }\KeywordTok{mutate}\NormalTok{(}\DataTypeTok{freq =}\NormalTok{ n}\OperatorTok{/}\KeywordTok{sum}\NormalTok{(n)) }\OperatorTok
\StringTok{  }\KeywordTok{filter}\NormalTok{(geschlecht }\OperatorTok{==}\StringTok{ "weiblich"}\NormalTok{) }\OperatorTok
\StringTok{  }\KeywordTok{arrange}\NormalTok{(}\OperatorTok{-}\NormalTok{freq) }\OperatorTok
\StringTok{  }\KeywordTok{mutate}\NormalTok{(}\DataTypeTok{freq =}\NormalTok{ scales}\OperatorTok{::}\KeywordTok{percent}\NormalTok{(freq))}
\end{Highlighting}
\end{Shaded}

\begin{verbatim}
## # A tibble: 8 x 4
## # Groups:   redner_fraktion [8]
##   geschlecht redner_fraktion           n freq 
##   <chr>      <chr>                 <int> <chr>
## 1 weiblich   BÜNDNIS 90/DIE GRÜNEN   407 61.4%
## 2 weiblich   fraktionslos             35 55.6%
## 3 weiblich   DIE LINKE               304 50.2%
## 4 weiblich   SPD                     367 38.5%
## 5 weiblich   <NA>                    202 36.4%
## 6 weiblich   FDP                     167 24.5%
## 7 weiblich   CDU/CSU                 202 14.9%
## 8 weiblich   AfD                      81 10.3%
\end{verbatim}

Der Anteil an Reden von Frauen ist sowohl bei der SPD, der CDU/CSU, den
Linken und auch der AfD niedriger als der Anteil an weiblichen
Abgeordneten. Besonders deutlich fällt dies aber bei der Fraktion
CDU/CSU aus. Bei der Fraktion der Grünen ist der Anteil an Reden von
Frauen hingegen sogar noch höher als der Anteil an weiblichen
Abgeordneten.

Die keiner Fraktion zugeordneten Reden stammen vor allem von
Regierungsmitgliedern. Nur bei Regierungsmitgliedern mit einem Mandat
für den Bundestag kann ein Geschlecht zugeordnet werden -- der Anteil an
Reden weiblicher Regierungsmitglieder fällt somit wohmöglich höher aus.

Der überraschend hohe Anteil an Reden weiblicher Abgeordneter bei den
Fraktionslosen abgeordneten geht auf Frauke Petry zurück, welche
besonders häufig am Podium des Bundestags zu sehen ist -- siehe
\ref{uberblick-der-reden}.

\hypertarget{literatur--und-quellverzeichnis}{%
\chapter*{Literatur- und
Quellverzeichnis}\label{literatur--und-quellverzeichnis}}
\addcontentsline{toc}{chapter}{Literatur- und Quellverzeichnis}

\hypertarget{refs}{}
\leavevmode\hypertarget{ref-back_2014}{}%
Bäck, Hanna/Debus, Marc/Müller, Jochen (2014): Who Takes the
Parliamentary Floor? The Role of Gender in Speech-Making in the Swedish
"Riksdag". In: \emph{Political Research Quarterly}, 67 (3), 504--518.

\leavevmode\hypertarget{ref-rcoreteam_2018}{}%
R Core Team (2018): R: A Language and Environment for Statistical
Computing. Vienna, Austria: R Foundation for Statistical Computing. Text
abrufbar unter: \url{https://www.R-project.org/}.

\leavevmode\hypertarget{ref-vaughan_2018}{}%
Vaughan, Davis/Dancho, Matt (2018): Furrr: Apply Mapping Functions in
Parallel Using Futures. Text abrufbar unter:
\url{https://CRAN.R-project.org/package=furrr}.

\leavevmode\hypertarget{ref-wangnerud_2000}{}%
Wangnerud, Lena (2000): Testing the Politics of Presence: Women's
Representation in the Swedish Riksdag. In: \emph{Scandinavian Political
Studies}, 23 (1), 67--91.

\leavevmode\hypertarget{ref-wickham_2016}{}%
Wickham, Hadley (2016): Rvest: Easily Harvest (Scrape) Web Pages. Text
abrufbar unter: \url{https://CRAN.R-project.org/package=rvest}.

\leavevmode\hypertarget{ref-wickham_2017}{}%
Wickham, Hadley (2017): Tidyverse: Easily Install and Load the
'Tidyverse'. Text abrufbar unter:
\url{https://CRAN.R-project.org/package=tidyverse}.

\leavevmode\hypertarget{ref-wickham_2018}{}%
Wickham, Hadley/Hester, James/Ooms, Jeroen (2018): Xml2: Parse XML. Text
abrufbar unter: \url{https://CRAN.R-project.org/package=xml2}.

\end{document}
